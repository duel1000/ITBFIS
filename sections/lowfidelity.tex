\subsection*{Udvikling af prototyper}
\addcontentsline{toc}{subsubsection}{Udvikling af prototyper}
Da vi allerede havde en ide om hvad vores produkt skulle kunne, startede vi vores udvikling af prototyperne ved hjælp af "scenario analysis" samt "task analysis", da vi havde en ide for hvad vi ville kunne med vores prototype, begyndte vi at lave en "mundtlig brainstorm", for at finde kunne frem til mulige løsninger. Dette resulterede at vi lavede 2 prototyper. Vi lavede efterfølgende "think-aloud" tests for at høre hvad der kunne gøres bedre eller hvad der var godt ved vores prototyper.

\subsection*{Test af prototyperne}
\addcontentsline{toc}{subsubsection}{Test af prototyperne}
Der var 2 personer til at lave Think-aloud test af Low-Fi prototyperne. De var begge førstegangs brugere. Da ingen af dem havde større kendskab til håndbold, var vi nød til at beskrive hvad applikationens hensigt var, samt at give dem en kort introduktion til domænet.\\\\ \textbf{Prototype 1, baggrundstanker:}
Ideen med denne prototype, var at den skulle benyttes på en touch-skærm, med en størrelse på ca 7"-10".(tablet-size)
Det var hensigten at bringe så meget af domænet ned i applikationen. og altså tegne banen op, og give brugeren mulighed for at sætte spilleren på sin plads på banen, i stedet for at vælge positionen fra de mulige (fløj, bak, center ol.)\\\\ \textbf{Beskrivelse af Prototypen:} Denne prototype blev udarbejdet ved brug af post-its og transparente A4-sider. Vi brugte hvid A4 papir, som baggrund. der var et ark papir, for hvert layout af baggrund,knapper og informationer.
Vi brugte transparent A4, til at symbolisere pop-up skærme, så det var tydeligt for test-personen at han ikke blev fjernet fra den foregående side.
og vi brugte post-its til de genstande, der skulle kunne flyttes rundt på.\\\\\textbf{Resultat af think-aloud test:} Begge test personer syntes at denne prototype var meget intuitiv. og det var kun en enkelt gang at den ene test person var i tvivl om hvordan en given task skulle løses. og de var derudover meget tilfredse over applikationen.
\\\\ \textbf{Prototype 2, baggrundstanker:} Denne prototype, var blev udviklet efter princippet "det bedste produkt, du har i hånden" (ok, den fandt jeg selv på :P) Men det ændre ikke  grunden til at vi ville teste om der ville være større tilfredshed med en mobilapplikation.
Denne prototype bygger på mange af de samme principper som den prototype 1. Da skærmen er meget mindre, valgte vi at fjerne "baggrunds billedet" og i stedet benytte tekst og knapper.\\\\ \textbf{Beskrivelse af prototypen:} for at lave en Low-Fi prototype klippede vi nogle stykker papir ud, på størrelse med en smartphone skærm, derefter tegnede vi de forskellige sider man kunne navigere rundt. De mindre stykker papir representerede "pop-ups" så man ikke troede at man skiftede billedet.\\\\ \textbf{Resultat af Think-aloud test:} Generelt syntes begge test personer at det hele var "for småt" (det mente vi dog skyldtes dårlig håndskrift). Derudover havde denne prototype langt større problem, fordi den stillede størrer krav til brugerens domæne kendskab. Derfor bliv vi nød til at hjælpe dem med at udfylde disse.
