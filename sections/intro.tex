Til mindre håndbold stævner bliver der spillet mange kampe, men der bliver ikke ført statistik for de enkelte spillere. Derfor burde det være muligt for forældrene eller dommeren at lave en live statistik over hele håndbold kampen. Derved vil det være muligt for forældrene at se deres børns udvikling.\\\\
Dette kan løses vha. applikationer til mobileenheder, hvor der er muligt at oprette holdene og derefter lave en statistik. 

\subsection*{Problemformulering}
\addcontentsline{toc}{subsubsection}{Problemformulering}
I dette projekt vil vi ved hjælp af forskellige prototyper prøve at finde den bedste løsning til problemet nævnt i introduktionen. Dette skal gøres skal hjælp af usability tests, heuristisk evaluering og analyse af målt data fra forsøgpersoner.

\subsection*{Scenarie}
\addcontentsline{toc}{subsubsection}{Scenarie}
En bruger ankommer til en håndboldturnering, og ønsker at føre statistik for en kamp. Derfor opretter brugeren holdene for den kommende kamp således, at applikationen er klar til at lave en statistik.\\\\Kamp opsætning:
\begin{itemize}
\item Hold 1
	\begin{itemize}
	\item Holdnavn
	\item Spillernavne
	\item Antal spillere
	\item Målmands navn
	\end{itemize}
\item Hold 2
	\begin{itemize}
	\item Holdnavn
	\item Spillernavne
	\item Antal spillere
	\item Målmands navn\\
	\end{itemize}
\end{itemize}
Når opsætning  er fuldført, kan kampen startes og brugeren kan begynde at lave statistik over kampen.\\\\
Liste over mulige begivenheder:
\begin{itemize}
	\item Hold X, Spiller X – Score
	\item Hold X, Spiller X – Saved
	\item Hold X, Spiller X – Missed
	\item Hold X, Spiller X – Blocked\\
\end{itemize}
Når kampen er afsluttet, kan brugeren få vist statistik over hele kampen.\\\\Da brugergruppen er forældre, er det vigtig, at applikationen har en nem brugergrænsefalde (usability). Til at løse usability, kan man benytte design konventioner, hvilket vil give brugeren et mere ensartet design, som ligner andre applikationer. For at kunne se user experience issues, er det muligt at lave nogle forskellige tests af applikationerne. Derved er det muligt at undersøge usability for applikationerne og for brugerne. 

   
   
