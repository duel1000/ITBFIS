Vi valgte ikke at bruge standardprojektet, men i stedet lave en applikation der handlede om håndboldstatistik. For at gøre det mere realistik oprettede vi to forskellige personaer, som projektet kunne være relevant for, og det gav os en retning at udviklede efter.\\\\I løbet af dette projekt har vi arbejdet med tre forskellige prototyper. Efter de første usability tests i projektet fandt vi hurtigt ud af, at prototype 1 nok var den bedste at gå efter, nummer 2 var interessant at følge og nummer 3 skulle smides væk efter samtale med de første testpersoner. Det viste sig også i de efterfølgende usability tests af de to prototyper at prototype 1 var den bedste. Både i binary success rate, time on task og satisfaction var prototype 1 den førende. Derfor i videre udvikling af projektet ville vi fokusere på prototype 1.\\\\Vi har lært at arbejde med forskellige metrics, og analyseret den data vi har fået ud af vores forskellige tests. Blandt andet kunne vi stille task success rate op for de to prototyper i forhold til hinanden, og se at prototype 1 umiddelbart var den bedeste. Da time on task målingerne lå så tæt på hinanden overlappede deres confidence interval, og det var derfor ikke muligt at fastslå om den ene prototype var bedre end den anden, men i de pågældende testmålinger havde prototype 1 stadig de bedste resultater.\\\\Udover dette har vi lavet en learnability curve for at se én brugers udvikling med gentagne gennemgange af testscenariet på prototype 1. 